\chapter{Analisi dei requisiti}
\label{cap:analisi-requisiti}

\intro{Breve introduzione al capitolo}\\

\section{Casi d'uso}

Per lo studio dei casi di utilizzo del prodotto sono stati creati dei diagrammi.
I diagrammi dei casi d'uso (in inglese \emph{Use Case Diagram}) sono diagrammi di tipo \gls{uml}\glsfirstoccur dedicati alla descrizione delle funzioni o servizi offerti da un sistema, così come sono percepiti e utilizzati dagli attori che interagiscono col sistema stesso.

\begin{usecase}{1}{Visualizzazione dashboard}
\usecaseactors{Utente}
\usecasepre{L'utente ha effettuato l'accesso al sistema}
\usecasedesc{L'utente visualizza la dashboard principale con le statistiche dei test eseguiti sui modelli di intelligenza artificiale generativa.}
\usecasepost{L'utente ha visualizzato una panoramica dei risultati dei test, inclusi grafici e tabelle riassuntive.}
\label{uc:uc1}
\end{usecase}

\begin{usecase}{2}{Visualizzazione statistiche}
\usecaseactors{Utente}
\usecasepre{L'utente sta visualizzando la dashboard}
\usecasedesc{L'utente visualizza le card con le statistiche aggregate dei test eseguiti sui modelli di intelligenza artificiale generativa.}
\usecasepost{L'utente ha visualizzato le statistiche aggregate dei test eseguiti, come il numero di test superati, il numero di test effettuati e altre metriche rilevanti.}
\label{uc:uc2}
\end{usecase}

\begin{usecase}{3}{Visualizzazione grafico temporale}
\usecaseactors{Utente}
\usecasepre{L'utente sta visualizzando la dashboard}
\usecasedesc{L'utente visualizza un grafico temporale con l'andamento dei test eseguiti sui modelli di intelligenza artificiale generativa.}
\usecasepost{L'utente ha visualizzato il grafico temporale con l'andamento dei test.}
\label{uc:uc3}
\end{usecase}

\begin{usecase}{4}{Visualizzazione tabella dei test}
\usecaseactors{Utente}
\usecasepre{L'utente sta visualizzando la dashboard}
\usecasedesc{L'utente visualizza una tabella con i test eseguiti sui target scelti durante la fase di testing}
\usecasepost{L'utente ha visualizzato la tabella con i test eseguiti sui target scelti durante la fase di testing.}
\label{uc:uc4}
\end{usecase}

\begin{figure}[!h] 
    \centering 
    \includegraphics[width=\textwidth]{usecase/1-4.png} 
    \caption{Use Case - UC1 to UC4}
\end{figure}

\begin{usecase}{5}{Visualizzazione singolo test}
\usecaseactors{Utente}
\usecasepre{L'utente ha selezionato un test dalla tabella dei test}
\usecasedesc{L'utente visualizza le informazioni dettagliate di un singolo test eseguito suil target di riferimento che comprende nome, scorer, dataset, target, status, coverage ed età del test.}
\usecasepost{L'utente ha visualizzato le informazioni dettagliate del singolo test selezionato.}
\label{uc:uc5}
\end{usecase}

\begin{usecase}{6}{Visualizzazione nome test}
\usecaseactors{Utente}
\usecasepre{L'utente sta visualizzando un singolo test}
\usecasedesc{L'utente visualizza il nome assegnato allo specifico test in esame} 
\usecasepost{L'utente ha visualizzato il nome del test selezionato.}
\label{uc:uc6}
\end{usecase}

\begin{usecase}{7}{Visualizzazione scorer test}
\usecaseactors{Utente}
\usecasepre{Precondizione da definire}
\usecasedesc{Descrizione da definire}
\usecasepost{Postcondizione da definire}
\label{uc:uc7}
\end{usecase}

\begin{usecase}{8}{Visualizzazione dataset test}
\usecaseactors{Utente}
\usecasepre{Precondizione da definire}
\usecasedesc{Descrizione da definire}
\usecasepost{Postcondizione da definire}
\label{uc:uc8}
\end{usecase}

\begin{usecase}{9}{Visualizzazione target test}
\usecaseactors{Utente}
\usecasepre{Precondizione da definire}
\usecasedesc{Descrizione da definire}
\usecasepost{Postcondizione da definire}
\label{uc:uc9}
\end{usecase}

\begin{usecase}{10}{Visualizzazione status test}
\usecaseactors{Utente}
\usecasepre{Precondizione da definire}
\usecasedesc{Descrizione da definire}
\usecasepost{Postcondizione da definire}
\label{uc:uc10}
\end{usecase}

\begin{usecase}{11}{Visualizzazione coverage test}
\usecaseactors{Utente}
\usecasepre{Precondizione da definire}
\usecasedesc{Descrizione da definire}
\usecasepost{Postcondizione da definire}
\label{uc:uc11}
\end{usecase}

\begin{usecase}{12}{Visualizzazione età test}
\usecaseactors{Utente}
\usecasepre{Precondizione da definire}
\usecasedesc{Descrizione da definire}
\usecasepost{Postcondizione da definire}
\label{uc:uc12}
\end{usecase}

\begin{usecase}{13}{Visualizzazione Status "Done"}
\usecaseactors{Utente}
\usecasepre{Precondizione da definire}
\usecasedesc{Descrizione da definire}
\usecasepost{Postcondizione da definire}
\label{uc:uc13}
\end{usecase}

\begin{usecase}{14}{Visualizzazione Status "Cancelled"}
\usecaseactors{Utente}
\usecasepre{Precondizione da definire}
\usecasedesc{Descrizione da definire}
\usecasepost{Postcondizione da definire}
\label{uc:uc14}
\end{usecase}

\begin{usecase}{15}{Visualizzazione Status "Failed"}
\usecaseactors{Utente}
\usecasepre{Precondizione da definire}
\usecasedesc{Descrizione da definire}
\usecasepost{Postcondizione da definire}
\label{uc:uc15}
\end{usecase}

\begin{figure}[!h] 
    \centering 
    \includegraphics[width=\textwidth]{usecase/4-15.png} 
    \caption{Use Case - UC4 to UC15}
\end{figure}
\newpage

\begin{usecase}{16}{}
\usecaseactors{Utente}
\usecasepre{Precondizione da definire}
\usecasedesc{Descrizione da definire}
\usecasepost{Postcondizione da definire}
\label{uc:uc16}
\end{usecase}

\begin{usecase}{17}{}
\usecaseactors{Utente}
\usecasepre{Precondizione da definire}
\usecasedesc{Descrizione da definire}
\usecasepost{Postcondizione da definire}
\label{uc:uc17}
\end{usecase}

\begin{usecase}{18}{}
\usecaseactors{Utente}
\usecasepre{Precondizione da definire}
\usecasedesc{Descrizione da definire}
\usecasepost{Postcondizione da definire}
\label{uc:uc18}
\end{usecase}

\begin{usecase}{19}{}
\usecaseactors{Utente}
\usecasepre{Precondizione da definire}
\usecasedesc{Descrizione da definire}
\usecasepost{Postcondizione da definire}
\label{uc:uc19}
\end{usecase}

\figure[!h] 
    \centering 
    \includegraphics[width=\textwidth]{usecase/16-19.png} 
    \caption{Use Case - UC16 to UC19}
\end{figure}

\begin{usecase}{20}{}
\usecaseactors{Utente}
\usecasepre{Precondizione da definire}
\usecasedesc{Descrizione da definire}
\usecasepost{Postcondizione da definire}
\label{uc:uc20}
\end{usecase}

\begin{usecase}{21}{}
\usecaseactors{Utente}
\usecasepre{Precondizione da definire}
\usecasedesc{Descrizione da definire}
\usecasepost{Postcondizione da definire}
\label{uc:uc21}
\end{usecase}

\begin{usecase}{22}{}
\usecaseactors{Utente}
\usecasepre{Precondizione da definire}
\usecasedesc{Descrizione da definire}
\usecasepost{Postcondizione da definire}
\label{uc:uc22}
\end{usecase}

\begin{usecase}{23}{}
\usecaseactors{Utente}
\usecasepre{Precondizione da definire}
\usecasedesc{Descrizione da definire}
\usecasepost{Postcondizione da definire}
\label{uc:uc23}
\end{usecase}

\begin{usecase}{24}{}
\usecaseactors{Utente}
\usecasepre{Precondizione da definire}
\usecasedesc{Descrizione da definire}
\usecasepost{Postcondizione da definire}
\label{uc:uc24}
\end{usecase}

\begin{usecase}{25}{}
\usecaseactors{Utente}
\usecasepre{Precondizione da definire}
\usecasedesc{Descrizione da definire}
\usecasepost{Postcondizione da definire}
\label{uc:uc25}
\end{usecase}

\begin{usecase}{26}{}
\usecaseactors{Utente}
\usecasepre{Precondizione da definire}
\usecasedesc{Descrizione da definire}
\usecasepost{Postcondizione da definire}
\label{uc:uc26}
\end{usecase}

\begin{usecase}{27}{}
\usecaseactors{Utente}
\usecasepre{Precondizione da definire}
\usecasedesc{Descrizione da definire}
\usecasepost{Postcondizione da definire}
\label{uc:uc27}
\end{usecase}

\begin{usecase}{28}{}
\usecaseactors{Utente}
\usecasepre{Precondizione da definire}
\usecasedesc{Descrizione da definire}
\usecasepost{Postcondizione da definire}
\label{uc:uc28}
\end{usecase}

\begin{usecase}{29}{}
\usecaseactors{Utente}
\usecasepre{Precondizione da definire}
\usecasedesc{Descrizione da definire}
\usecasepost{Postcondizione da definire}
\label{uc:uc29}
\end{usecase}




\section{Tracciamento dei requisiti}

Da un'attenta analisi dei requisiti e degli \emph{use case} effettuata sul progetto è stata stilata la tabella che traccia i requisiti in rapporto agli \emph{use case}.\\
Sono stati individuati diversi tipi di requisiti e si è quindi fatto utilizzo di un codice identificativo per distinguerli.\\
Il codice dei requisiti è così strutturato R(F/Q/V)(N/D/O) dove:
\begin{enumerate}
	\item[R =] requisito
    \item[F =] funzionale
    \item[Q =] qualitativo
    \item[V =] di vincolo
    \item[N =] obbligatorio (necessario)
    \item[D =] desiderabile
    \item[O =] opzionale
\end{enumerate}
Nelle tabelle \ref{tab:requisiti-funzionali}, \ref{tab:requisiti-qualitativi} e \ref{tab:requisiti-vincolo} sono riassunti i requisiti e il loro tracciamento con gli use case delineati in fase di analisi.

\newpage

\begin{table}%
\caption{Tabella del tracciamento dei requisti funzionali}
\label{tab:requisiti-funzionali}
\begin{tabularx}{\textwidth}{lXl}
\hline\hline
\textbf{Requisito} & \textbf{Descrizione} & \textbf{Use Case}\\
\hline
RFN-1     & L'interfaccia permette di configurare il tipo di sonde del test & UC1 \\
\hline
\end{tabularx}
\end{table}%

\begin{table}%
\caption{Tabella del tracciamento dei requisiti qualitativi}
\label{tab:requisiti-qualitativi}
\begin{tabularx}{\textwidth}{lXl}
\hline\hline
\textbf{Requisito} & \textbf{Descrizione} & \textbf{Use Case}\\
\hline
RQD-1    & Le prestazioni del simulatore hardware deve garantire la giusta esecuzione dei test e non la generazione di falsi negativi & - \\
\hline
\end{tabularx}
\end{table}%

\begin{table}%
\caption{Tabella del tracciamento dei requisiti di vincolo}
\label{tab:requisiti-vincolo}
\begin{tabularx}{\textwidth}{lXl}
\hline\hline
\textbf{Requisito} & \textbf{Descrizione} & \textbf{Use Case}\\
\hline
RVO-1    & La libreria per l'esecuzione dei test automatici deve essere riutilizzabile & - \\
\hline
\end{tabularx}
\end{table}%
