\chapter{Introduzione}
\label{cap:introduzione}

Il presente capitolo introduce il contesto applicativo oggetto dello stage: automazione di testing su applicazioni che sfruttano l'intelligenza artificiale generativa, come chatbot o assistenti virtuali. Queste applicazioni trattano input testuali e informazioni sensibili, esponendo a rischi generali quali manipolazioni degli input, fuoriuscite di dati e comportamenti imprevisti dei modelli.\\
Il progetto mira a definire e implementare un approccio integrato per valutare e migliorare la sicurezza, l'affidabilità e la conformità delle applicazioni, attraverso analisi del codice, test e pratiche operative dedicate.\\
% \gls{api}. \\

% \noindent Esempio di citazione in linea \\
% \cite{site:agile-manifesto}. \\

% \noindent Esempio di citazione nel pie' di pagina \\
% citazione\footcite{womak:lean-thinking} \\

\section{L'azienda}
L'azienda con cui ho deciso di intraprendere il mio percorso di Stage è Var Group S.p.A., 
una realtà italiana specializzata in soluzioni IT e servizi di consulenza per le imprese.\\
La scelta è stata influenzata dalla mia precedente esperienza con il progetto di Software Engineering svolto presso Var Group,
durante il quale ho avuto modo di apprezzare la professionalità e la competenza del team.


\section{Introduzione al caso di studio}
Negli ultimi anni, l'intelligenza artificiale generativa ha rivoluzionato il modo in cui le applicazioni interagiscono con gli utenti, offrendo esperienze più naturali e coinvolgenti.
Questa rivoluzione ha cambiato e continua a cambiare il modo in cui gli utenti e le aziende si interfacciano alle applicazioni software.\\
Una considerevole porzione delle aziende di tutto il mondo e di tutti gli ambiti hanno integrato o stanno integrando modelli di intelligenza artificiale nei propri prodotti.

\subsection{Necessità del testing per AI generativa}
Tuttavia l'adozione di questa innovativa tecnologia viene con nuove sfide da affrontare per quanto riguarda la sicurezza e la privacy.\\
Le applicazioni basate su AI generativa sono esposte a rischi specifici inesistenti prima della loro introduzione, come attacchi di \emph{prompt injection}, \emph{data leakage} e \emph{allucinazioni} dei modelli. Questi rischi possono compromettere la sicurezza delle applicazioni, la privacy degli utenti e l'integrità dei dati trattati.\\
Per riuscire a mitigare i rischi associati all'utilizzo di modelli di intelligenza artificiale generativa, è fondamentale seguire delle linea guida specifiche come quelle fornite da OWASP, che ha stilato una lista delle top 10 vulnerabilità legate all'utilizzo dell'intelligenza artificiale generativa.\\
Il progetto di stage si concentra proprio su questo aspetto, il testing delle applicazioni che integrano una LLM per verificarne la conformità con le linee guida OWASP in modo tale da scovare le vulnerabilità e prevenirne l'utilizzo da parte di utenti malintenzionati.\\

\section{Software da sviluppare}



\subsection{L'idea}

\newline

Riguardo la stesura del testo, relativamente al documento sono state adottate le seguenti convenzioni tipografiche:
\begin{itemize}
	\item gli acronimi, le abbreviazioni e i termini ambigui o di uso non comune menzionati vengono definiti nel glossario, situato alla fine del presente documento;
	\item per la prima occorrenza dei termini riportati nel glossario viene utilizzata la seguente nomenclatura: \emph{parola}\glsfirstoccur;
	\item i termini in lingua straniera o facenti parti del gergo tecnico sono evidenziati con il carattere \emph{corsivo}.
\end{itemize}
