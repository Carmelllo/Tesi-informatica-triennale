\chapter{Introduzione}
\label{cap:introduzione}

Il presente capitolo introduce il contesto applicativo oggetto dello stage: automazione di testing su applicazioni che sfruttano l'intelligenza artificiale generativa, come chatbot o assistenti virtuali. Queste applicazioni trattano input testuali e informazioni sensibili, esponendo a rischi generali quali manipolazioni degli input, fuoriuscite di dati e comportamenti imprevisti dei modelli.\\
Il progetto mira a definire e implementare un approccio integrato per valutare e migliorare la sicurezza, l'affidabilità e la conformità delle applicazioni, attraverso analisi del codice, test e pratiche operative dedicate.\\
% \gls{api}. \\

% \noindent Esempio di citazione in linea \\
% \cite{site:agile-manifesto}. \\

% \noindent Esempio di citazione nel pie' di pagina \\
% citazione\footcite{womak:lean-thinking} \\

\section{L'azienda}
L'azienda con cui ho deciso di intraprendere il mio percorso di Stage è Var Group S.p.A., 
una realtà italiana specializzata in soluzioni IT e servizi di consulenza per le imprese.\\
La scelta è stata influenzata dalla mia precedente esperienza con il progetto di Software Engineering svolto presso Var Group,
durante il quale ho avuto modo di apprezzare la professionalità e la competenza del team.

\section{L'idea}
Il progetto di stage è incentrato sulla sicurezza delle applicazioni basate su AI generativa, con particolare attenzione all'impiego di analisi statica e dinamica del codice per verificarne la conformità alle linee guida OWASP. \\
L'obiettivo è sviluppare un sistema di testing e verifica capace di rilevare e prevenire vulnerabilità specifiche (come le tecniche di prompt injection) evitando la divulgazione di informazioni riservate a utenti non autorizzati e impedendo l'introduzione di falle nell'applicativo.\\
Il lavoro comprende inoltre la progettazione e l'integrazione di workflow e strumenti automatizzati per l'analisi del flusso dei dati, il fuzzing e i penetration test, nonché la definizione di procedure di mitigazione e di reporting delle vulnerabilità individuate.

\section{Organizzazione del testo}

\begin{description}
    \item[{\hyperref[cap:processi-metodologie]{Il secondo capitolo}}] descrive ...
    
    \item[{\hyperref[cap:descrizione-stage]{Il terzo capitolo}}] descrive lo stage e le attività svolte.
    
    \item[{\hyperref[cap:analisi-requisiti]{Il quarto capitolo}}]  ...
    
    \item[{\hyperref[cap:progettazione-codifica]{Il quinto capitolo}}]  ...
    
    \item[{\hyperref[cap:verifica-validazione]{Il sesto capitolo}}]  ...
    
    \item[{\hyperref[cap:conclusioni]{Nel settimo capitolo}}]  ...
\end{description}

Riguardo la stesura del testo, relativamente al documento sono state adottate le seguenti convenzioni tipografiche:
\begin{itemize}
	\item gli acronimi, le abbreviazioni e i termini ambigui o di uso non comune menzionati vengono definiti nel glossario, situato alla fine del presente documento;
	\item per la prima occorrenza dei termini riportati nel glossario viene utilizzata la seguente nomenclatura: \emph{parola}\glsfirstoccur;
	\item i termini in lingua straniera o facenti parti del gergo tecnico sono evidenziati con il carattere \emph{corsivo}.
\end{itemize}
