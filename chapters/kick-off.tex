\chapter{Descrizione dello stage}
\label{cap:descrizione-stage}

\intro{Breve introduzione al capitolo}\\

\section{Introduzione al progetto}


\section{Analisi preventiva dei rischi}

Durante la fase iniziale di analisi sono stati identificati i principali rischi potenziali connessi al progetto, classificati per ambito (tecnico, di progetto e infrastrutturale) e priorizzati in base all'impatto e alla probabilità. Per ciascun rischio è stato predisposto un piano di mitigazione che definisce azioni concrete, tempistiche e responsabilità precise.

Le contromisure prevedono attività di sperimentazione controllata degli strumenti, revisioni manuali dei risultati, integrazione e test in ambienti rappresentativi, oltre a piani di escalation per le criticità più gravi. È inoltre prevista una procedura di monitoraggio continuo e revisione periodica delle valutazioni e delle soluzioni adottate, in modo da aggiornare rapidamente le contromisure alla luce di nuovi dati o evoluzioni tecnologiche.

\subsection{Rischi tecnici}

\begin{risk}{Complessità nell'applicare strumenti di security testing ad AI generativa (tool immaturi o
non sempre affidabili).}
    \riskdescription{Le difficoltà nell'adattare i tool di security testing all'AI generativa possono derivare dalla loro immaturità o dalla mancanza di affidabilità}
    \risksolution{Una lunga fase di sperimentazione e testing dei tool mitigherà i rischi, garantendo risultati affidabili}
\end{risk}

\begin{risk}{Possibili falsi positivi o negativi nei test di vulnerabilità.}
    \riskdescription{I test di vulnerabilità potrebbero generare risultati inaccurati, con falsi positivi (segnalazioni errate di vulnerabilità) o falsi negativi (mancata rilevazione di vulnerabilità reali)}
    \risksolution{Implementare una fase di revisione manuale dei risultati dei test per convalidare le segnalazioni e ridurre il rischio di falsi positivi e negativi}
\end{risk}

\begin{risk}{Difficoltà di integrazione dei tool con codice reale e pipeline di sviluppo.}
    \riskdescription{Le difficoltà di integrazione possono derivare da incompatibilità tra i tool di testing e l'infrastruttura esistente, nonché dalla complessità del codice reale su cui si stanno eseguendo i test}
    \risksolution{Collaborare con gli sviluppatori del codice reale per garantire che i tool di testing siano compatibili con l'infrastruttura esistente e fornire supporto durante l'integrazione}
\end{risk}

\subsection{Rischi di progetto}

\begin{risk}{Mancanza di esperienza pregressa su OWASP o sicurezza AI.}
    \riskdescription{La poca familiarità con le best practices di OWASP o con le specificità della sicurezza nell'AI generativa potrebbe rallentare l'avanzamento del progetto}
    \risksolution{Studio e formazione con risorse adeguate per aumentare la familiarità con OWASP e la sicurezza dell'AI generativa}
\end{risk}

\begin{risk}{Possibile difficoltà a rispettare la pianificazione a causa della curva di apprendimento iniziale.}
    \riskdescription{La curva di apprendimento iniziale per l'utilizzo di nuovi strumenti e tecnologie potrebbe richiedere più tempo del previsto, influenzando la pianificazione del progetto}
    \risksolution{Pianificare margini di tempo aggiuntivi per la formazione e l'adattamento agli strumenti, nonché monitorare attentamente i progressi}
\end{risk}

\subsection{Rischi infrastrutturali}

\begin{risk}{Limitazioni di risorse computazionali nei test di AI.}
    \riskdescription{Le risorse computazionali disponibili per l'esecuzione dei test di AI potrebbero non essere sufficienti, causando rallentamenti o interruzioni nei test}
    \risksolution{Ottimizzare l'uso delle risorse disponibili e, se necessario, richiedere l'accesso a risorse computazionali aggiuntive}
\end{risk}

\begin{risk}{Problemi di compatibilità con ambienti cloud o di deployment.}
    \riskdescription{Le differenze tra gli ambienti di sviluppo e produzione potrebbero causare problemi di compatibilità, rendendo difficile l'esecuzione dei test in modo uniforme}
    \risksolution{Testare i tool di testing in ambienti simili a quelli di produzione e documentare eventuali problemi di compatibilità}
\end{risk}


\section{Requisiti e obiettivi}


\subsection{Obiettivi obbligatori}
\begin{itemize}
\item Valutazione comparativa degli strumenti di analisi.
\item Applicazione pratica dei test su codice reale.
\item Prototipo in grado di generare report sulle vulnerabilità AI rispetto a OWASP.
\item Documentazione tecnica e presentazione finale.
\end{itemize}

\subsection{Obiettivi desiderabili}
\begin{itemize}
\item Dashboard interattiva con visualizzazioni avanzate
\item Integrazione del prototipo in pipeline CI/CD esistente.
\item Estensione dei test ad altri framework oltre Gandalf Test.
\item Raccomandazioni per un framework interno di AI Security by Design.
\end{itemize}

\section{Pianificazione}

La pianificazione del lavoro di progetto è stata suddivisa in fasi settimanali, con obiettivi specifici per ciascuna fase. Di seguito è riportata una panoramica della pianificazione prevista:

\begin{table}[htbp]
    \centering
    \renewcommand{\arraystretch}{1.2}
    \begin{tabular}{|p{3cm}|p{10cm}|}
        \hline
        \textbf{Settimana} & \textbf{Attività} \\
        \hline
        Settimana 1 & Studio preliminare su OWASP e rischi AI, overview di Gandalf Test, setup ambiente di
lavoro.\\
        \hline
        Settimana 2 & Analisi comparativa di tool di analisi statica e dinamica (open-source e commerciali).
Creazione matrice di valutazione.\\
        \hline
        Settimana 3 & Applicazione degli strumenti a piccoli progetti demo, valutazione dei risultati e raccolta
criticità.\\
        \hline
        Settimana 4 & Esecuzione dei primi test su componenti reali del team, documentazione dei risultati,
identificazione vulnerabilità.\\
        \hline
        Settimana 5 & Realizzazione di script/report per aggregare risultati, definizione dei KPI di compliance
OWASP.\\
        \hline
        Settimana 6 & Sviluppo di dashboard interattiva per monitorare vulnerabilità e andamento dei test.\\
        \hline
        Settimana 7 & Test end-to-end sul prototipo, miglioramento dei tool e dei report.\newline\mbox{}\\
        \hline
        Settimana 8 & Redazione di documentazione tecnica, manuale utente e materiale per la presentazione
della tesi.\\
        \hline
    \end{tabular}
    \caption{Pianificazione delle attività di progetto}
\end{table}