% Acronyms
\newacronym[description={\glslink{apig}{Application Program Interface}}]
    {api}{API}{Application Program Interface}

\newacronym[description={\glslink{umlg}{Unified Modeling Language}}]
    {uml}{UML}{Unified Modeling Language}

% (duplicate removed)

\newacronym[description={\glslink{owaspg}{Open Web Application Security Project}}]
    {owasp}{OWASP}{Open Web Application Security Project}

\newacronym[description={\glslink{cicdg}{Continuous Integration/Continuous Deployment}}]
    {cicd}{CI/CD}{Continuous Integration/Continuous Deployment}


% Glossary entries
\newglossaryentry{apig} {
    name=\glslink{api}{API},
    text=Application Program Interface,
    sort=api,
    description={in informatica con il termine \emph{Application Programming Interface API} (ing. interfaccia di programmazione di un'applicazione) si indica ogni insieme di procedure disponibili al programmatore, di solito raggruppate a formare un set di strumenti specifici per l'espletamento di un determinato compito all'interno di un certo programma. La finalità è ottenere un'astrazione, di solito tra l'hardware e il programmatore o tra software a basso e quello ad alto livello semplificando così il lavoro di programmazione}
}

\newglossaryentry{umlg} {
    name=\glslink{uml}{UML},
    text=UML,
    sort=uml,
    description={in ingegneria del software \emph{UML, Unified Modeling Language} (ing. linguaggio di modellazione unificato) è un linguaggio di modellazione e specifica basato sul paradigma object-oriented. L'\emph{UML} svolge un'importantissima funzione di ``lingua franca'' nella comunità della progettazione e programmazione a oggetti. Gran parte della letteratura di settore usa tale linguaggio per descrivere soluzioni analitiche e progettuali in modo sintetico e comprensibile a un vasto pubblico}
}

% Placeholder glossary entries (setup only). Descriptions left as 'Da definire.' to be filled later.
\newglossaryentry{ai-generativa}{
    name={AI generativa},
    text={AI generativa},
    sort=ai-generativa,
    description={il termine \emph{AI generativa} si riferisce a una categoria di modelli di intelligenza artificiale progettati per generare nuovi contenuti, come testo, immagini o video. Questi modelli sono addestrati su grandi quantità di dati prodotti dall'uomo e utilizzano tecniche avanzate di apprendimento automatico per creare output che possono essere simili a quelli creati dagli esseri umani.}
}

\newglossaryentry{llm}{
    name={LLM},
    text={LLM},
    sort=llm,
    description={in informatica, un \emph{Large Language Model (LLM)} è un modello di intelligenza artificiale addestrato su un enorme quantità di testo e codice con la capacità di comprendere, generare e tradurre il linguaggio umano}
}

\newglossaryentry{owaspg}{
    name=\glslink{owasp}{OWASP},
    text={OWASP},
    sort=owasp,
    description={\emph{OWASP} è una fondazione no profit che si occupa di migliorare la sicurezza del software. Offre risorse e linee guida per la sicurezza delle applicazioni web e di sistemi che integrano componenti di intelligenza artificiale (tra cui liste di vulnerabilità e best practice).}
}

\newglossaryentry{gandalf-test}{
    name={Gandalf Test},
    text={Gandalf Test},
    sort=gandalf-test,
    description={il \emph{Gandalf Test} è un gioco creato da Lakera che permette all'utente di entrare nel mondo del testing delle LLM fornendo vari livelli di difficoltà incrementale in cui l'obiettivo è estrapolare una password da un LLM vulnerabile.}
}

\newglossaryentry{prompt-injection}{
    name={Prompt injection},
    text={Prompt injection},
    sort=prompt-injection,
    description={la \emph{prompt injection} è una tecnica di attacco in cui un attaccante manipola l'input fornito a un modello di linguaggio per indurlo a generare output dannosi o indesiderati. Questo tipo di attacco può compromettere la sicurezza e l'affidabilità dei modelli di intelligenza artificiale generativa.}
}

\newglossaryentry{data-leakage}{
    name={Data leakage},
    text={Data leakage},
    sort=data-leakage,
    description={il \emph{data leakage} si riferisce alla situazione in cui informazioni sensibili o riservate vengono inavvertitamente esposte o divulgate a persone non autorizzate, spesso a causa di vulnerabilità nei sistemi di intelligenza artificiale generativa.}
}

\newglossaryentry{allucinazioni}{
    name={Allucinazioni},
    text={Allucinazioni},
    sort=allucinazioni,
    description={la \emph{allucinazione} si riferisce a un fenomeno in cui un modello di intelligenza artificiale genera output che sono errati, fuorvianti o privi di fondamento nella realtà. Questo può accadere quando il modello interpreta in modo errato l'input o quando i dati di addestramento contengono informazioni imprecise.}
}

\newglossaryentry{analisi-statica}{
    name={Analisi statica},
    text={Analisi statica},
    sort=analisi-statica,
    description={la \emph{analisi statica} è una tecnica di verifica del software che analizza il codice sorgente senza eseguirlo, al fine di identificare potenziali vulnerabilità, errori o violazioni delle best practice.}
}

\newglossaryentry{analisi-dinamica}{
    name={Analisi dinamica},
    text={Analisi dinamica},
    sort=analisi-dinamica,
    description={la \emph{analisi dinamica} è una tecnica di verifica del software che analizza il comportamento del codice durante l'esecuzione, al fine di identificare vulnerabilità, errori o violazioni delle best practice.}
}

\newglossaryentry{red-teaming}{
    name={Red teaming},
    text={Red teaming},
    sort=red-teaming,
    description={il \emph{red teaming} è una pratica di sicurezza informatica in cui un team di esperti / uno strumento simula attacchi informatici per identificare vulnerabilità e punti deboli nei sistemi di difesa di un'organizzazione.}
}

\newglossaryentry{jailbreak}{
    name={Jailbreak },
    text={Jailbreak},
    sort=jailbreak,
    description={Da definire.}
}

\newglossaryentry{cicdg}{
    name=\glslink{cicd}{CI/CD},
    text={CI/CD},
    sort=cicd,
    description={La \emph{CI/CD} (Continuous Integration/Continuous Deployment) è una pratica di sviluppo che automatizza l'integrazione del codice e il rilascio, migliorando la qualità del software e riducendo i tempi di consegna.}
}

\newglossaryentry{pipeline}{
    name={Pipeline},
    text={Pipeline},
    sort=pipeline,
    description={Da definire.}
}

\newglossaryentry{sdk}{
    name={SDK},
    text={SDK},
    sort=sdk,
    description={Da definire.}
}

\newglossaryentry{dashboard}{
    name={Dashboard},
    text={Dashboard},
    sort=dashboard,
    description={Da definire.}
}

\newglossaryentry{kpi}{
    name={KPI},
    text={KPI},
    sort=kpi,
    description={Da definire.}
}

\newglossaryentry{bias}{
    name={Bias},
    text={Bias},
    sort=bias,
    description={Da definire.}
}

\newglossaryentry{dataset}{
    name={Dataset},
    text={Dataset},
    sort=dataset,
    description={Un \emph{dataset} è una raccolta di dati utilizzata per addestrare, testare e valutare modelli di intelligenza artificiale. I dataset possono variare in dimensione, formato e contenuto, e la loro qualità è fondamentale per il successo dei modelli di machine learning.}
}

\newglossaryentry{open-source}{
    name={Open source},
    text={open source},     
    sort=opensource,
    description={Modello di sviluppo e distribuzione del software in cui il codice sorgente è reso disponibile al pubblico e può essere utilizzato, modificato e ridistribuito secondo i termini della relativa licenza.}
}

\newglossaryentry{token}{
    name={Token},
    text={Token},
    sort=token,
    description={Da definire.}
}


% Tools (skeletons)
\newglossaryentry{promptfoo}{
    name={PromptFoo},
    text={PromptFoo},
    sort=promptfoo,
    description={}
}

\newglossaryentry{pyrit}{
    name={PyRIT},
    text={PyRIT},
    sort=pyrit,
    description={}
}

\newglossaryentry{langfuse}{
    name={LangFuse},
    text={LangFuse},
    sort=langfuse,
    description={Da definire.}
}

\newglossaryentry{deepeval}{
    name={DeepEval / DeepTeam},
    text={DeepEval / DeepTeam},
    sort=deepeval,
    description={Da definire.}
}

\newglossaryentry{garak}{
    name={Garak},
    text={Garak},
    sort=garak,
    description={Da definire.}
}

\newglossaryentry{giskard}{
    name={Giskard},
    text={Giskard},
    sort=giskard,
    description={Da definire.}
}

\newglossaryentry{galileo}{
    name={Galileo},
    text={Galileo},
    sort=galileo,
    description={Da definire.}
}

\newglossaryentry{lakeraguard}{
    name={LakeraGuard},
    text={LakeraGuard},
    sort=lakeraguard,
    description={Da definire.}
}

\newglossaryentry{use-case}{
    name={Caso d'uso},
    text={Caso d'uso},
    sort=use-case,
    description={Da definire.}
}

\newglossaryentry{matrice-valutazione}{
    name={Matrice di valutazione},
    text={Matrice di valutazione},
    sort=matrice-valutazione,
    description={Da definire.}
}

\newglossaryentry{prototipo}{
    name={Prototipo},
    text={Prototipo},
    sort=prototipo,
    description={Un \emph{prototipo} è una versione preliminare di un prodotto, utilizzata per testare e convalidare idee, concetti o funzionalità prima dello sviluppo finale. I prototipi possono variare in fedeltà e complessità, da schizzi su carta a modelli interattivi ad alta fedeltà.}
}