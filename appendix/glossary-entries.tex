% Acronyms
\newacronym[description={\glslink{apig}{Application Program Interface}}]
    {api}{API}{Application Program Interface}

\newacronym[description={\glslink{umlg}{Unified Modeling Language}}]
    {uml}{UML}{Unified Modeling Language}

\newacronym[description={\glslink{owaspg}{Open Web Application Security Project}}]
    {owasp}{OWASP}{Open Web Application Security Project}

\newacronym[description={\glslink{cicdg}{Continuous Integration/Continuous Deployment}}]
    {cicd}{CI/CD}{Continuous Integration/Continuous Deployment}

\newacronym[description={\glslink{iidg}{Iterative and Incremental Development}}]
    {iid}{IID}{Iterative and Incremental Development}

\newacronym[description={\glslink{dodg}{United States Department of Defense}}]
    {dod}{DoD}{Department of Defense}

\newacronym[description={\glslink{erpg}{Enterprise Resource Planning}}]
    {erp}{ERP}{Enterprise Resource Planning}

% Glossary entries
\newglossaryentry{apig} {
    name=\glslink{api}{API},
    text=Application Program Interface,
    sort=api,
    description={in informatica con il termine \emph{Application Programming Interface API} (ing. interfaccia di programmazione di un'applicazione) si indica ogni insieme di procedure disponibili al programmatore, di solito raggruppate a formare un set di strumenti specifici per l'espletamento di un determinato compito all'interno di un certo programma. La finalità è ottenere un'astrazione, di solito tra l'hardware e il programmatore o tra software a basso e quello ad alto livello semplificando così il lavoro di programmazione}
}

\newglossaryentry{umlg} {
    name=\glslink{uml}{UML},
    text=UML,
    sort=uml,
    description={in ingegneria del software \emph{UML, Unified Modeling Language} (ing. linguaggio di modellazione unificato) è un linguaggio di modellazione e specifica basato sul paradigma object-oriented. L'\emph{UML} svolge un'importantissima funzione di ``lingua franca'' nella comunità della progettazione e programmazione a oggetti. Gran parte della letteratura di settore usa tale linguaggio per descrivere soluzioni analitiche e progettuali in modo sintetico e comprensibile a un vasto pubblico}
}

\newglossaryentry{iidg} {
    name=\glslink{iid}{IID},
    text=IID,
    sort=iid,
    description={lo sviluppo \emph{Iterative and Incremental Development (IID)} è una metodologia di sviluppo software che combina la progettazione iterativa con un modello incrementale di costruzione. Il prodotto viene sviluppato attraverso cicli ripetuti (iterazioni) e in porzioni più piccole (incrementi), permettendo agli sviluppatori di sfruttare ciò che è stato appreso durante lo sviluppo delle versioni precedenti e di adattarsi ai cambiamenti dei requisiti.}
}

\newglossaryentry{dodg} {
    name=\glslink{dod}{DoD},
    text=DoD,
    sort=dod,
    description={il \emph{Department of Defense (DoD)} è il Dipartimento della Difesa degli Stati Uniti d'America, responsabile della sicurezza nazionale e delle forze armate statunitensi. Nel contesto dello sviluppo software, il DoD ha storicamente promosso e standardizzato metodologie di sviluppo iterativo-incrementale per i propri progetti.}
}

\newglossaryentry{slack} {
    name={Slack},
    text={Slack},
    sort=slack,
    description={\emph{Slack} è una piattaforma di comunicazione aziendale che offre funzionalità di messaggistica istantanea, canali tematici, condivisione di file e integrazione con altri strumenti di produttività. È ampiamente utilizzata per la collaborazione in team di sviluppo software.}
}

% Placeholder glossary entries (setup only). Descriptions left as 'Da definire.' to be filled later.
\newglossaryentry{chatbot}{
    name={Chatbot},
    text={chatbot},
    sort=chatbot,
    description={un \emph{chatbot} è un programma informatico progettato per simulare una conversazione con utenti umani, tipicamente attraverso interfacce testuali o vocali. I chatbot moderni spesso integrano modelli di intelligenza artificiale generativa per fornire risposte più naturali e contestuali.}
}

\newglossaryentry{ai-generativa}{
    name={AI generativa},
    text={AI generativa},
    sort=ai-generativa,
    description={il termine \emph{AI generativa} si riferisce a una categoria di modelli di intelligenza artificiale progettati per generare nuovi contenuti, come testo, immagini o video. Questi modelli sono addestrati su grandi quantità di dati prodotti dall'uomo e utilizzano tecniche avanzate di apprendimento automatico per creare output che possono essere simili a quelli creati dagli esseri umani.}
}

\newglossaryentry{llm}{
    name={LLM},
    text={LLM},
    sort=llm,
    description={in informatica, un \emph{Large Language Model (LLM)} è un modello di intelligenza artificiale addestrato su un enorme quantità di testo e codice con la capacità di comprendere, generare e tradurre il linguaggio umano}
}

\newglossaryentry{owaspg}{
    name=\glslink{owasp}{OWASP},
    text={OWASP},
    sort=owasp,
    description={\emph{OWASP} è una fondazione no profit che si occupa di migliorare la sicurezza del software. Offre risorse e linee guida per la sicurezza delle applicazioni web e di sistemi che integrano componenti di intelligenza artificiale (tra cui liste di vulnerabilità e best practice).}
}

\newglossaryentry{gandalf-test}{
    name={Gandalf Test},
    text={Gandalf Test},
    sort=gandalf-test,
    description={il \emph{Gandalf Test} è un gioco creato da Lakera che permette all'utente di entrare nel mondo del testing delle LLM fornendo vari livelli di difficoltà incrementale in cui l'obiettivo è estrapolare una password da un LLM vulnerabile.}
}

\newglossaryentry{prompt-injection}{
    name={Prompt injection},
    text={Prompt injection},
    sort=prompt-injection,
    description={la \emph{prompt injection} è una tecnica di attacco in cui un attaccante manipola l'input fornito a un modello di linguaggio per indurlo a generare output dannosi o indesiderati. Questo tipo di attacco può compromettere la sicurezza e l'affidabilità dei modelli di intelligenza artificiale generativa.}
}

\newglossaryentry{data-leakage}{
    name={Data leakage},
    text={Data leakage},
    sort=data-leakage,
    description={il \emph{data leakage} si riferisce alla situazione in cui informazioni sensibili o riservate vengono inavvertitamente esposte o divulgate a persone non autorizzate, spesso a causa di vulnerabilità nei sistemi di intelligenza artificiale generativa.}
}

\newglossaryentry{allucinazioni}{
    name={Allucinazioni},
    text={Allucinazioni},
    sort=allucinazioni,
    description={la \emph{allucinazione} si riferisce a un fenomeno in cui un modello di intelligenza artificiale genera output che sono errati, fuorvianti o privi di fondamento nella realtà. Questo può accadere quando il modello interpreta in modo errato l'input o quando i dati di addestramento contengono informazioni imprecise.}
}

\newglossaryentry{erpg} {
    name=\glslink{erp}{ERP},
    text=ERP,
    sort=erp,
    description={l'\emph{Enterprise Resource Planning (ERP)} è un sistema software integrato che consente alle aziende di gestire e automatizzare i principali processi di business, come contabilità, gestione delle risorse umane, produzione, supply chain e vendite, attraverso un'unica piattaforma centralizzata.}
}

\newglossaryentry{guardrail}{
    name={Guardrail},
    text={guardrail},
    sort=guardrail,
    description={nel contesto dell'intelligenza artificiale, i \emph{guardrail} sono meccanismi di sicurezza e controllo implementati per limitare o prevenire comportamenti indesiderati dei modelli di AI. Includono filtri sui contenuti, limiti sugli argomenti trattabili e sistemi di moderazione automatica delle risposte.}
}

\newglossaryentry{analisi-statica}{
    name={Analisi statica},
    text={Analisi statica},
    sort=analisi-statica,
    description={la \emph{analisi statica} è una tecnica di verifica del software che analizza il codice sorgente senza eseguirlo, al fine di identificare potenziali vulnerabilità, errori o violazioni delle best practice.}
}

\newglossaryentry{analisi-dinamica}{
    name={Analisi dinamica},
    text={Analisi dinamica},
    sort=analisi-dinamica,
    description={la \emph{analisi dinamica} è una tecnica di verifica del software che analizza il comportamento del codice durante l'esecuzione, al fine di identificare vulnerabilità, errori o violazioni delle best practice.}
}

\newglossaryentry{red-teaming}{
    name={Red teaming},
    text={Red teaming},
    sort=red-teaming,
    description={il \emph{red teaming} è una pratica di sicurezza informatica in cui un team di esperti / uno strumento simula attacchi informatici per identificare vulnerabilità e punti deboli nei sistemi di difesa di un'organizzazione.}
}

\newglossaryentry{jailbreak}{
    name={Jailbreak},
    text={jailbreak},
    sort=jailbreak,
    description={Nel contesto dell'intelligenza artificiale, il \emph{jailbreak} è una tecnica utilizzata per aggirare le restrizioni e i filtri di sicurezza implementati in un modello di linguaggio, inducendolo a generare contenuti che normalmente sarebbero bloccati o censurati.}
}

\newglossaryentry{cicdg}{
    name=\glslink{cicd}{CI/CD},
    text={CI/CD},
    sort=cicd,
    description={La \emph{CI/CD} (Continuous Integration/Continuous Deployment) è una pratica di sviluppo che automatizza l'integrazione del codice e il rilascio, migliorando la qualità del software e riducendo i tempi di consegna.}
}

\newglossaryentry{pipeline}{
    name={Pipeline},
    text={pipeline},
    sort=pipeline,
    description={Una \emph{pipeline} è una sequenza automatizzata di processi o fasi che trasformano un input in un output finale. Nel contesto dello sviluppo software, indica tipicamente l'insieme di operazioni automatiche di build, test e deploy.}
}

\newglossaryentry{sdk}{
    name={SDK},
    text={SDK},
    sort=sdk,
    description={Un \emph{Software Development Kit (SDK)} è un insieme di strumenti, librerie, documentazione e codice di esempio che consente agli sviluppatori di creare applicazioni per una specifica piattaforma o servizio.}
}

\newglossaryentry{dashboard}{
    name={Dashboard},
    text={dashboard},
    sort=dashboard,
    description={Una \emph{dashboard} è un'interfaccia grafica che presenta in modo visuale e sintetico informazioni, metriche e indicatori chiave di performance (KPI), permettendo agli utenti di monitorare lo stato di un sistema o processo.}
}

\newglossaryentry{kpi}{
    name={KPI},
    text={KPI},
    sort=kpi,
    description={I \emph{Key Performance Indicator (KPI)} sono metriche quantificabili utilizzate per valutare il successo di un'organizzazione, progetto o attività nel raggiungimento degli obiettivi prefissati.}
}

\newglossaryentry{dataset}{
    name={Dataset},
    text={Dataset},
    sort=dataset,
    description={Un \emph{dataset} è una raccolta di dati utilizzata per addestrare, testare e valutare modelli di intelligenza artificiale. I dataset possono variare in dimensione, formato e contenuto, e la loro qualità è fondamentale per il successo dei modelli di machine learning.}
}

\newglossaryentry{open-source}{
    name={Open source},
    text={open source},     
    sort=opensource,
    description={Modello di sviluppo e distribuzione del software in cui il codice sorgente è reso disponibile al pubblico e può essere utilizzato, modificato e ridistribuito secondo i termini della relativa licenza.}
}

\newglossaryentry{token}{
    name={Token},
    text={token},
    sort=token,
    description={Nel contesto dei modelli di linguaggio, un \emph{token} è l'unità minima di testo elaborata dal modello. Può corrispondere a una parola, parte di una parola, un carattere o un simbolo di punteggiatura.}
}


% Tools (skeletons)
\newglossaryentry{promptfoo}{
    name={PromptFoo},
    text={PromptFoo},
    sort=promptfoo,
    description={\emph{PromptFoo} è una piattaforma open source per il testing e la valutazione di modelli di linguaggio, con supporto per test automatizzati, integrazione CI/CD e analisi delle vulnerabilità OWASP per AI generativa.}
}

\newglossaryentry{pyrit}{
    name={PyRIT},
    text={PyRIT},
    sort=pyrit,
    description={\emph{PyRIT (Python Risk Identification Tool)} è uno strumento open source sviluppato da Microsoft Azure per l'identificazione e la mitigazione dei rischi di sicurezza nei modelli di intelligenza artificiale generativa.}
}

\newglossaryentry{langfuse}{
    name={LangFuse},
    text={LangFuse},
    sort=langfuse,
    description={\emph{LangFuse} è una piattaforma open source per l'osservabilità e la valutazione di applicazioni basate su modelli di linguaggio, con funzionalità di tracciamento, analisi e debugging.}
}

\newglossaryentry{deepeval}{
    name={DeepEval / DeepTeam},
    text={DeepEval},
    sort=deepeval,
    description={\emph{DeepEval} è un framework open source per la valutazione e il benchmarking di modelli di AI generativa. \emph{DeepTeam} è il relativo strumento di red teaming per testare la sicurezza dei modelli.}
}

\newglossaryentry{garak}{
    name={Garak},
    text={Garak},
    sort=garak,
    description={\emph{Garak} è uno scanner di vulnerabilità open source sviluppato da NVIDIA per testare modelli di AI generativa, focalizzato su attacchi come prompt injection, jailbreak e data leakage.}
}

\newglossaryentry{giskard}{
    name={Giskard},
    text={Giskard},
    sort=giskard,
    description={\emph{Giskard} è una piattaforma di red teaming automatizzato per testare e analizzare modelli di AI, disponibile sia come servizio cloud a pagamento che come libreria Python open source.}
}

\newglossaryentry{galileo}{
    name={Galileo},
    text={Galileo},
    sort=galileo,
    description={\emph{Galileo} è una piattaforma commerciale per la valutazione e l'osservabilità di applicazioni basate su AI generativa, con SDK disponibili in Python e TypeScript.}
}

\newglossaryentry{lakeraguard}{
    name={LakeraGuard},
    text={LakeraGuard},
    sort=lakeraguard,
    description={\emph{LakeraGuard} è una piattaforma commerciale di sicurezza per modelli di AI sviluppata da Lakera, che offre scansione delle vulnerabilità, monitoraggio delle minacce e protezione in tempo reale.}
}

\newglossaryentry{use-case}{
    name={Caso d'uso},
    text={caso d'uso},
    sort=use-case,
    description={Un \emph{caso d'uso} (o \emph{use case}) è una descrizione di come un utente interagisce con un sistema per raggiungere un obiettivo specifico. Nell'ingegneria del software viene utilizzato per definire i requisiti funzionali.}
}

\newglossaryentry{matrice-valutazione}{
    name={Matrice di valutazione},
    text={matrice di valutazione},
    sort=matrice-valutazione,
    description={Una \emph{matrice di valutazione} è uno strumento analitico che permette di confrontare diverse opzioni o alternative sulla base di criteri predefiniti, assegnando punteggi per facilitare la scelta della soluzione migliore.}
}

\newglossaryentry{prototipo}{
    name={Prototipo},
    text={Prototipo},
    sort=prototipo,
    description={Un \emph{prototipo} è una versione preliminare di un prodotto, utilizzata per testare e convalidare idee, concetti o funzionalità prima dello sviluppo finale. I prototipi possono variare in fedeltà e complessità, da schizzi su carta a modelli interattivi ad alta fedeltà.}
}

\newglossaryentry{backend}{
    name={Backend},
    text={backend},
    sort=backend,
    description={Il \emph{backend} è la parte di un'applicazione software che gestisce la logica di business, l'elaborazione dei dati e la comunicazione con il database. Opera lato server e non è direttamente visibile all'utente finale.}
}

\newglossaryentry{frontend}{
    name={Frontend},
    text={frontend},
    sort=frontend,
    description={Il \emph{frontend} è la parte di un'applicazione software con cui l'utente interagisce direttamente. Include l'interfaccia utente, il design grafico e tutti gli elementi visuali dell'applicazione.}
}

\newglossaryentry{saas}{
    name={SaaS},
    text={SaaS},
    sort=saas,
    description={\emph{Software as a Service (SaaS)} è un modello di distribuzione del software in cui le applicazioni sono ospitate da un provider di servizi e rese disponibili agli utenti tramite Internet, senza necessità di installazione locale.}
}

\newglossaryentry{data-poisoning}{
    name={Data poisoning},
    text={data poisoning},
    sort=data-poisoning,
    description={Il \emph{data poisoning} è un tipo di attacco informatico in cui dati malevoli vengono introdotti nel dataset di addestramento di un modello di machine learning per comprometterne il comportamento e le prestazioni.}
}

\newglossaryentry{black-box}{
    name={Black-box},
    text={black-box},
    sort=black-box,
    description={Il \emph{testing black-box} è una metodologia di test in cui il tester non ha accesso al codice sorgente o alla struttura interna del sistema. I test vengono eseguiti basandosi esclusivamente sugli input e output del sistema.}
}

\newglossaryentry{compliance}{
    name={Compliance},
    text={compliance},
    sort=compliance,
    description={La \emph{compliance} indica la conformità a normative, standard, regolamenti e best practice applicabili a un determinato settore o ambito tecnologico.}
}